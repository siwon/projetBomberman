\section*{Présentation du projet}\addcontentsline{toc}{section}{Présentation du projet}
Ce document présente la conception choisie pour le programme. Celui-ci est découpé en 5 modules distincts pouvant être développés séparément. Chacun de ces modules est décrit ci-après, et est accompagné de diagrammes UML permettant de représenter sa structure interne, et externe, ainsi que son mode de fonctionnement.

\vspace{0.5cm}

\subsection{Bibliothèques utilisées}

Ce programme sera réalisé à l'aide du langage \textbf{C++}. La bibliothèque graphique utilisée sera la \textbf{SFML 1.6} choisie pour sa simplicité et son efficacité. Celle-ci est sous licence zlib/png qui permet son utilisation sans aucune contreparties.

Dans ce rapport seront présentés les tests réalisés lors du rendu au client du programme terminé. Ces tests permettront de déterminer si le jeu fonctionne correctement.

\section{Les tests}

Les différentes fonctionnalitées (réseau, graphique, etc ...) seront testées séparément.

\subsection{Les tests réseau}

Ces tests permettront de déterminer si le jeu est bien jouable en réseau.

Le premier test sera bien entendu de faire fonctionner le jeu sur un unique ordinateur.

Si ce test s'avère concluant, nous pourrons ensuite essayer d'ajouter un ordinateur et de vérifier que l'on puisse joueur puis d'ajouter encore deux autres ordinateurs au réseau afin de vérifier qu'avec un maximum de joueur, tous sur un ordinateur différent, le jeu fonctionne correctement.

\subsection{Les tests graphique}

Ces tests permettront de vérifier que l'interface graphique fonctionne correctement.

Ces tests sont simples, il suffira de lancer le jeu, de tester les différents choix du menu, ainsi que de tester l'affichage durant la phase de jeu.

\subsection{Les tests de contrôleurs de jeu}

Ces tests permettront de vérifier que les contrôleurs de jeu permettent de jouer au jeu.



\subsection{Les tests de jouabilité}

Ces tests permettront de déterminé si le jeu dans sa globalité (gestion des parties, du multijoueurs, etc ...) fonctionne bien.

Ces tests de jouabilité consisteront à faire plusieurs parties et de vérifier que le jeu fonctionne bien. Pour cela, deux joueurs sur un même ordinateur suffiront à déterminer cela.


\section*{Conclusion}\addcontentsline{toc}{section}{Conclusion}
Cette description de chaque module est suivie en annexe des diagrammes de classes correspondants, ainsi que du diagramme de composants et de déploiement du programme.\\

Cette conception est susceptible d'évoluer au cours du développement du programme. Elle donne juste une vue globale de la structure du projet.\\

Pour les diagrammes de classes, la légende des couleurs est la suivante :

\begin{itemize}
    \item \textbf{Bleu} : Classe
    \item \textbf{Vert} : Interface fournie
    \item \textbf{Orange} : Interface requise
    \item \textbf{Jaune} : Structure
    \item \textbf{Violet} : Enumération
    \item \textbf{Gris} : Classe déjà implémentée
\end{itemize}


La liste de test ci-dessus n'est pas exaustive. Des tests pourront donc être ajoutés dans le futur.
