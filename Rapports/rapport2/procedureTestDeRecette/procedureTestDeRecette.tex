\section*{Présentation du projet}\addcontentsline{toc}{section}{Présentation du projet}
	Ce document fournit la procédure d'installation de PolyBomber$^{\mbox{\scriptsize{\copyright}}}$, veuillez suivre attentivement les différentes étapes afin de profiter pleinement de votre jeu.

Dans ce rapport seront présentés les tests réalisés lors du rendu au client du programme terminé. Ces tests permettront de déterminer si le jeu fonctionne correctement.

\section{Les tests}

Les différentes fonctionnalitées (réseau, graphique, etc ...) seront testées séparément.

\subsection{Les tests réseau}

Ces tests permettront de déterminer si le jeu est bien jouable en réseau.

Le premier test sera bien entendu de faire fonctionner le jeu sur un unique ordinateur.

Si ce test s'avère concluant, nous pourrons ensuite essayer d'ajouter un ordinateur et de vérifier que l'on puisse joueur puis d'ajouter encore deux autres ordinateurs au réseau afin de vérifier qu'avec un maximum de joueur, tous sur un ordinateur différent, le jeu fonctionne correctement.

\subsection{Les tests graphique}

Ces tests permettront de vérifier que l'interface graphique fonctionne correctement.

Ces tests sont simples, il suffira de lancer le jeu, de tester les différents choix du menu, ainsi que de tester l'affichage durant la phase de jeu.

\subsection{Les tests de contrôleurs de jeu}

Ces tests permettront de vérifier que les contrôleurs de jeu permettent de jouer au jeu.



\subsection{Les tests de jouabilité}

Ces tests permettront de déterminé si le jeu dans sa globalité (gestion des parties, du multijoueurs, etc ...) fonctionne bien.

Ces tests de jouabilité consisteront à faire plusieurs parties et de vérifier que le jeu fonctionne bien. Pour cela, deux joueurs sur un même ordinateur suffiront à déterminer cela.


\section*{Conclusion}\addcontentsline{toc}{section}{Conclusion}
Ce cahier des charges fonctionnel a décrit de manière aussi exhaustive que possible les différentes fonctionnalités du programme. 

Toutes les prochaines étapes de conception, modélisation et programmation du programme se baseront sur ce document. 

Toutefois, il se peut que des ambiguités persistent, c'est pourquoi ce document n'est pas figé et sera éventuellement soumis à des modifications.

La liste de test ci-dessus n'est pas exaustive. Des tests pourront donc être ajoutés dans le futur.
