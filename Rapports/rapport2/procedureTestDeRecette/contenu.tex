\section*{Introduction}\addcontentsline{toc}{section}{Introduction}

Dans ce rapport seront présentés les tests réalisés lors du rendu au client du programme terminé. Ces tests permettront de déterminer si le jeu fonctionne correctement.

\vspace{0.5cm}

Les différentes fonctionnalitées (réseau, graphique, etc ...) seront testées séparément.

%\newpage

\section{Les conditions de tests}

%dans quelles conditions seront réalisés les tests

\subsection{Les ordinateurs}

Les tests seront effectués sur les ordinateurs de la salle réseau. La distribution utilisée sera la distribution linux.

\subsection{Le réseau}

Les tests supposent que les ordinateurs soient déjà paramétrés pour fonctionner en réseau local.

%\newpage

\section{Les tests}

\subsection{Tests du réseau}

Ces tests permettront de déterminer si le jeu est bien jouable en réseau.

Le premier test sera bien entendu de faire fonctionner le jeu sur un unique ordinateur.

Le second test consistera à ajouter un ordinateur et de vérifier que l'on puisse jouer en réseau.

Le test suivant consistera à ajouter encore deux autres ordinateurs afin de vérifier qu'avec un maximum de joueur, tous sur un ordinateur différent, le jeu fonctionne correctement.

\subsection{Tests d'interface graphique}

Ces tests permettront de vérifier que l'interface graphique fonctionne correctement.

Le premier test d'interface graphique consistera à naviguer dans les menus afin de vérifier que leur affichage fonctionne.

Le test suivant consistera simplement à jouer et de terminer une partie afin de tester l'affichage final.

\subsection{Tests de contrôleurs de jeu}

Ces tests permettront de vérifier que les contrôleurs de jeu permettent de jouer au jeu. De plus ils permettront de vérifier le bon fonctionnement du paramétrage des commandes.

Un premier test consistera à utiliser un clavier pour deux.

Le second test consistera à utiliser un clavier pour quatre joueurs. Ceci permettra de vérifier si toutes les interactions des touches du clavier fonctionnent bien.

Le troisième test consistera à utiliser un joystick. Ceci permettra de tester la configuration ainsi que le paramétrage des touches.

Le quatrième test consistera à utiliser la Wiimote. Ceci permettra de tester la configuration ainsi que le paramétrage des touches.

\subsection{Tests de jouabilité}

Ces tests permettront de déterminé si le jeu dans sa globalité (gestion des parties, du multijoueurs, etc ...) fonctionne bien.

Ces tests de jouabilité consisteront à faire plusieurs parties et de vérifier que le jeu fonctionne bien. Pour cela, deux joueurs sur un même ordinateur utilisant un clavier suffiront à déterminer cela.

%\newpage

\section*{Conclusion}\addcontentsline{toc}{section}{Conclusion}

La liste de test ci-dessus n'est pas exhaustive. Des tests pourront donc être ajoutés dans le futur.
