\subsection{Présentation de l'interface graphique}

Lorsque le programme démarrera, une fenêtre d'une taille de 800 pixels de largeur sur 600 pixels de hauteur apparaîtra représentant pendant quelques secondes l'écran d'accueil du programme, c'est-à-dire une image occupant tout la fenêtre nommant le programme et présentant éventuellement les noms des créateurs. Un fondu laissera ensuite apparaître le menu principal.

\subsection{Menus}

Pour chaque menu présenté ci-dessous, les options affichées seront sélectionnables grâce aux touches directionnelles haut et bas du clavier. L'option  actuellement sélectionnée sera représentée d'une manière différente, en changeant sa couleur, ou en affichant un curseur devant elle. La validation de l'option sélectionnée s'effectuera par pression de la touche \emph{Entrée}. Un appui sur la touche \emph{Echap} permettra de revenir sur l'ecran de menu précédent.

\subsubsection{Menu principal}

Le menu principal présentera trois options à l'utilisateur : \emph{Jouer}, \emph{Options}, \emph{Quitter}, permettant respectivement d'aller sur le menu de jeu, de paramétrage, ou de quitter le programme sans confirmation. Le nom du jeu sera également visible au-dessus de ces options. Sur ce menu, la touche \emph{Echap} n'aura pas d'effet.

\subsubsection{Menus de paramétrage}

Ce menu permettra à l'utilisateur de pouvoir régler les options générales du programme. Il présentera une option \emph{Contrôles} permettant de configurer les contrôleurs et les touches de jeu, comme défini dans le prochain paragraphe. Il affichera en plus une option \emph{Audio} permettant d'aller sur le menu de configuration audio. Enfin, la dernière option \emph{Graphisme} permettra de changer les options graphiques du jeu en allant sur le menu du même nom. Une option \emph{Retour} pour retourner au menu principal sera également affichée.

\paragraph{Menu de configuration des contrôleurs}

Ce menu permettra de gérer l'attribution des touches pour chaque joueur. Un premier ecran listera les quatre joueurs en précisant le type de contrôleur utilisé. Une option \emph{Retour} pour retourner au menu de paramétrage sera également affichée. Lorsque l'utilisateur sélectionnera un joueur, un second ecran s'affichera. Celui-ci présentera en haut de la fenêtre un widget de sélection d'un contrôleur (clavier, joystick ou Wiimote) où le nom du contrôleur est affichée entourée de deux flèches permettant de sélectionner le contrôleur précédent ou suivant de l'actuel avec les flèches directionnelles gauche et droite. En dessous de ce widget se trouvera soit une image représentant le contrôleur avec l'emplacement des boutons prédéfinis pour la Wiimote, soit une liste des actions possibles avec l'assignation de la touche à côté. Lorsque l'action sera sélectionnée puis validée, le programme attendra que l'utilisateur saisisse la touche pour l'assigner. Lorsque l'utilisateur voudra assigner une touche déjà assignée, le programme affichera un message d'erreur informant l'utilisateur que la touche est déjà assignée. Sur cet ecran, un bouton de retour vers le menu qui liste les joueurs.

\paragraph{Menu de configuration audio}

Ce menu gérera le volume de la musique et des effets sonores du jeu et du menu avec deux widgets permettant à l'utilisateur de choisir le volume entre 0\% et 100\%. Un bouton permettant de couper directement le son sera présent à côté de chacun des widgets. Un bouton de retour au menu précédent sera également visible.

\paragraph{Menu de configuration graphique}

Ce menu permettra de configurer l'aspect graphique du jeu. Il présentera une option permettant de passer du mode fenêtré au mode plein écran. En-dessous se trouvera un widget permettant de choisir le pack de textures à utiliser pour le programme. Celui-ci listera les packs disponibles et l'utilisateur pourra en sélectionner un. Un bouton permettant de revenir au menu de paramétrage sera aussi présent.

\subsubsection{Menus de jeu}

Cet enchaînement de menus va permettre à l'utilisateur de jouer en configurant la partie. Le premier écran sera composé de deux options, \emph{Créer une partie} et \emph{Rejoindre une partie}. Un bouton permettant de revenir au menu principal sera aussi présent, tout comme dans les sous-menus décrits ci-dessous.

\paragraph{Rejoindre une partie}

Ce menu permet de jouer une partie sur un réseau local. Un premier écran permettra à l'utilisateur d'entrer une adresse IP correspondant à l'ordinateur hôte de la partie dans une zone de texte. Lorsque le joueur validera sa saisie, soit un message d'erreur s'affichera si un problème est survenu lors de la connexion (adresse incorrecte, partie non lancée, etc...), soit un nouvel écran permettra de choisir le nombre de joueurs à ajouter à la partie en fonction des places disponibles. Un nom pourra leur être attribué (un nom par défaut sera attribué sinon). Une fois cet écran validé, un second apparaîtra afin d'attendre le début de la partie.

\paragraph{Créer une partie}

Ce menu permet de créer une nouvelle partie, qu'elle soit en réseau ou locale. Une première fenêtre permettra de définir le nombre de joueurs jouant sur l'ordinateur courant ainsi que leur nom. Des boutons permettant de définir si la partie se joue en réseau, et de configurer les options de la partie seront également présents. L'écran suivant permettra de confirmer les personnes présentes dans la partie en indiquant leur personnage, nom et score. Si la partie se joue en réseau, l'adresse IP de l'ordinateur sera visible. Un bouton permettra de lancer la partie.

\paragraph{Options de la partie}

Cet écran listera chaque bonus pouvant être présent dans le jeu, et un widget présent à côté de chacun d'eux permettra de choisir leur nombre pendant la partie, en augmentant ou réduisant le nombre par défaut. Un bouton permettant de choisir un nombre aléatoire sera aussi présent. En dessous de ce tableau se trouvera un bouton permettant de revenir à l'écran de configuration de la partie.

\subsection{Plateau de jeu}

La fenêtre du plateau de jeu sera composée de la carte du jeu (contenant 19 blocs carrés de largeur et 13 blocs carrés de hauteur) sur laquelle évolueront les personnages et les objets (bombes et options) du jeu. En-dessous de la carte seront affichés les scores du match actuel avec le nom du joueur aux couleurs de son personnage.\\

Un appui sur la touche \emph{Echap} de l'un des joueurs mettra en pause le jeu et affichera un menu permettant de configurer les options en appelant le menu de paramétrage, de revenir au menu principal, ou de quitter directement le programme, après confirmation. Lorsqu'un joueur quitte une partie, celle-ci se termine si la partie est gérée par l'ordinateur abandonnant. Le joueur est simplement retiré de la partie sinon.\\

\subsection{Fin de partie}

Lorsqu'une partie se termine, un premier écran s'affiche, désignant soit le vainqueur de la partie si un des personnages a gagné, soit informant le match nul. Un bouton permettra de passer à l'écran suivant qui correspondra à l'écran visible juste avant de lancer la première partie, et décrit ci-dessus, affichant les scores de chacun et permettant de relancer la partie pour l'ordinateur hôte de la partie. Un bouton additionnel sera également présent pour recommencer un match, qui affichera un écran récapitulant les scores du match avant de le relancer.