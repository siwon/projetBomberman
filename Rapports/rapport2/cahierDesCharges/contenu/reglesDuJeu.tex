\subsection{But du jeu}

Le but principal du jeu Bomberman est d'être le dernier survivant de la partie. Des joueurs sont répartis sur une carte et doivent déposer des bombes de façon stratégique afin d'éliminer les autres joueurs en les faisant exploser.

Le jeu se joue sous forme de match composé d'un nombre inconnu de parties. A la fin de chaque partie, l'ordinateur ayant créé la partie choisit de continuer ou de recommencer un match. Chaque partie rapporte un point à son vainqueur (le dernier à rester en vie). Le match est gagné par le joueur qui a accumulé le plus de points au cours des parties.

\subsection{Nombre de joueurs}

Le nombre de joueurs doit être d'un minimum de 2 et d'un maximum de 4. Ces joueurs peuvent jouer sur un ou plusieurs ordinateurs en réseau.

\subsection{Déroulement d'une partie}

\subsubsection{Situation initiale}

La carte est une carte rectangulaire de 19 cases horizontales et 13 cases verticales. Chaque case est remplie par une entité de jeu contenue dans cette liste (lors d'un début de partie, les cases ne peuvent être remplie que par les 4 premières entités) :

\begin{itemize}
    \item Vide
    \item Personnage
    \item Mur indestructible
    \item Caisse à exploser
    \item Bombe
    \item Bonus
    \item Déflagration de bombe
\end{itemize}

Les joueurs sont placés selon le nombre de joueurs : 
\begin{itemize}
\item \textbf{4 joueurs} : Les joueurs sont placés dans les 4 coins de la carte.
\item \textbf{3 joueurs} : 2 joueurs sont placés dans les coins supérieurs et le troisième est placé sur la dernière ligne au centre.
\item \textbf{2 joueurs} : Un joueur se trouve dans le coin supérieur gauche, le deuxième dans le coin inférieur droit.
\end{itemize}

Au début, chaque joueur sera dans une zone avec au moins 3 cases vides consécutives afin de pouvoir faire exploser au moins une bombe. De plus, les joueurs ne peuvent poser qu'une bombe à la fois, se déplacent à vitesse normale et les bombes ont une portée de \porteeParDefaut .

\subsubsection{Déroulement}

Les joueurs ainsi que des caisses sont réparties sur la carte selon le nombre de joueurs et les paramètres. Ensuite, la partie démarre. Les joueurs peuvent poser des bombes afin de détruire des caisses. À chaque destruction de caisse, il peut y avoir un bonus qui apparaît. Si la déflagration d'une bombe touche un joueur (que se soit son poseur ou un autre joueur), ce joueur est considéré comme mort et disparaît de la carte. Tant qu'il reste des caisses, la portée des bombes dépend des bonus pris par le bomberman. Une fois toutes les caisses détruites, la portée des bombes devient maximale et les déflagrations traversent toute la carte.

\subsubsection{Fin de partie}

La partie est terminée lorsqu'il n'y a plus qu'un survivant sur la carte. Dans le cas où il n'y aurait pas de survivant (les derniers joueurs meurent à la suite de l'explosion des dernières bombes), la partie est considérée comme nulle.

\subsection{Bonus - Malus}

Il existe différents types de bonus. Certains améliorent ou bien dégradent les caractéristiques des bombes des joueurs, d'autres celles des personnages. Il existe aussi des "infections" qui sont des malus qui affectent un ou plusieurs joueurs.

\subsubsection{Bonus pour les bombes}

Les bonus et malus disponibles pour les bombes seront :
\begin{itemize}
\item \textbf{Bombe Up} : Augmentation du nombre de bombes pouvant être posées simultanément par un joueur
\item \textbf{Bombe Down} : Diminution du nombre de bombes pouvant être posées simultanément par un joueur
\item \textbf{Flamme Jaune} : Augmentation de la portée d'une bombe d'une case
\item \textbf{Flamme Bleue} : Diminution de la portée d'une bombe d'une case
\item \textbf{Flamme Rouge} : Augmentation de la portée de la bombe au maximum
\item \textbf{Mine} : Bombe qui explose lorsqu'un joueur est à portée
\item \textbf{Bombe à pics} : Bombe dont la déflagration ne s'arrêtent pas lorsqu'elle rencontre une caisse
\item \textbf{Bombe atomique} : Explosion circulaire qui explose en 8-connexité d'une portée celle du joueur
\end{itemize}

\subsubsection{Bonus pour les personnages}

Les bonus et malus disponibles pour les personnages seront :
\begin{itemize}
\item \textbf{Patins} : Augmentation de la vitesse de déplacement du personnage
\item \textbf{Sabots} : Diminution de la vitesse de déplacement du personnage
\item \textbf{Ligne de bombes} : Une option qui permet au personnage de poser toutes ses bombes d'un seul coup, alignées devant lui,
\item \textbf{Détonateur} : Ce bonus permet de transformer toutes les bombes posées en bombes télécommandées.
\end{itemize}

\subsubsection{Infections}

Les infections disponibles seront :
\begin{itemize}
\item \textbf{Crâne} : Infection aléatoire parmi toutes les infections,
\item \textbf{Devil} : Tous les bomberman sont touchés par une maladie aléatoire, y compris celui qui a pris le Devil,
\item \textbf{Confusion} : Inversion des touches du clavier,
\item \textbf{Spasmes} : Mouvement vers une case tous les \nbSecondes secondes,
\item \textbf{Dilatation} : Lenteur du personnage au maximum,
\item \textbf{Fureur} : Pose une bombe toutes les \nbSecondes secondes.
\end{itemize}

\subsubsection{Probabilités des bonus - malus}

\begin{figure}[h]
	\begin{center}
		\begin{tabular}{|r|c|}
		\hline 
		& Probabilités \\ 
		\hline 
		Bonus pour les bombes & \% \\ 
		\hline 
		Bonus pour le personnage & \% \\ 
		\hline 
		Infections & \% \\ 
		\hline 
		\end{tabular} 
	\end{center}
	\caption{Probabilité des catégories}
\end{figure}

\begin{figure}[h]
	\begin{center}
		\begin{tabular}{|r|c|}
\hline 
& Probabilités \\ 
\hline 
Bombe Up & • \\ 
\hline 
Bombe Down & • \\ 
\hline 
Flamme Jaune & • \\ 
\hline 
Flamme Bleue & • \\ 
\hline 
Flamme Rouge & • \\ 
\hline 
Mine & • \\ 
\hline 
Bombe à pics & • \\ 
\hline 
Bombe atomique & • \\ 
\hline 
\end{tabular} 
	\end{center}
	\caption{Probabilité des bonus pour les bombes}
\end{figure}

\begin{figure}[h]
	\begin{center}
		\begin{tabular}{|r|c|}
\hline 
& Probabilités \\ 
\hline 
Patins & • \\ 
\hline 
Sabots & • \\ 
\hline 
Ligne de bombes & • \\ 
\hline 
Détonateur & • \\ 
\hline 
\end{tabular} 
	\end{center}
	\caption{Probabilité des bonus pour les personnages}
\end{figure}

\begin{figure}[h]
	\begin{center}
		\begin{tabular}{|r|c|}
\hline 
& Probabilités \\ 
\hline 
Crâne & • \\ 
\hline 
Devil & • \\ 
\hline 
Confusion & • \\ 
\hline 
Spasmes & • \\ 
\hline 
Dilatation & • \\ 
\hline 
Fureur & • \\ 
\hline 
\end{tabular} 
	\end{center}
	\caption{Probabilité des infections}
\end{figure}