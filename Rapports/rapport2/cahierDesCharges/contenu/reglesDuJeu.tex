Les règles du jeu du Bomberman sont simples.

\subsection{But du jeu}

Le but principal du jeu Bomberman est d'être le dernier survivant de la partie. Des joueurs sont réparties sur une carte et doivent déposer des bombes stratégiquement afin d'éliminer les autres joueurs.

Chaque manche rapporte des points qui sont comptabilisés par match afin de déterminer un vainqueur.

\subsection{Nombre de joueurs}

Le nombre de joueur doit être d'un minimum de 2 et d'un maximum de 4. Ces joueurs peuvent jouer sur un ou plusieurs ordinateurs via le réseau.

\subsection{Situation initiale}

La carte est une carte rectangulaire de 19 cases horizontales et 13 cases verticales.

Les joueurs sont placés selon le nombre de joueurs : 
\begin{itemize}
\item \textbf{4 joueurs} : Les joueurs sont placés dans les 4 coins de la carte.
\item \textbf{3 joueurs} : 2 joueurs sont placés dans les coins supérieurs et le troisième est placé sur la dernière ligne au centre.
\item \textbf{2 joueurs} : Un joueur se trouve dans le coin supérieur gauche, le deuxième dans le coin inférieur droit.
\end{itemize}

\subsection{Déroulement}

Les joueurs 

\subsection{Fin de partie}

La partie est terminée lorsqu'il n'y a plus qu'un survivant sur la carte. Dans le cas où il n'y aurait pas de survivant (les derniers joueurs meurt à la suite de l'explosion des dernières bombes), la partie est considérée comme nulle.

\subsection{Bonus}

Il existe différents types de bonus, les "bonus de bombe" qui améliorent ou bien dégradent les caractéristiques des bombes des joueurs, les "bonus de bomberman" celle du bomberman, et les "infections" qui sont des malus qui affectent un ou plusieurs joueurs.

\subsubsection{Bonus de bombe}

Les bonus de bombe disponibles seront :
\begin{itemize}
\item \textbf{Bombe Up} : Augmentation du nombre de bombes,
\item \textbf{Bombe Down} : Diminution du nombre de bombes,
\item \textbf{Flamme Jaune} : augmentation de la portée d'une bombe,
\item \textbf{Flamme Bleue} : Diminution de la portée d'une bombe,
\item \textbf{Flamme Rouge} : Portée de la bombe au maximum,
\item \textbf{Mine} : Une bombe qui explose lorsqu'un joueur est à portée,
\item \textbf{Spike Bomb} : Cette bombe ornée de pointes possède une explosion capable de passer à travers les blocs destructibles, permettant ainsi de détruire plusieurs blocs alignés ou de toucher des adversaires situés derrière des blocs mous,
\item \textbf{H Bomb} : Explosion circulaire qui explose en 8 connexité d'une portée celle du joueur.
\end{itemize}

\subsubsection{Bonus de bomberman}

Les bonus de bomberman disponibles seront :
\begin{itemize}
\item \textbf{Patins} : Augmentation de la vitesse de déplacement du bomberman,
\item \textbf{Sabots} : Diminution de la vitesse de déplacement du bomberman,
\item \textbf{Ligne de bombe} : Une option qui permet au bomberman de poser toutes ses bombes d'un seul coup, alignées devant lui,
\item \textbf{Détonateur} : Ce bonus permet de transformer toutes les bombes posées en bombes télécommandées.
\end{itemize}

\subsubsection{Infections}

Les infections disponibles seront :
\begin{itemize}
\item \textbf{Crâne} : Infection aléatoire parmi toutes les infections,
%\item \textbf{Ebola} : 3 infections aléatoires parmis toutes les infections,
\item \textbf{Devil} : Tous les bomberman sont touchés par une maladie aléatoire, y compris celui qui a pris le Devil,
\item \textbf{Confusion} : Inversion des touches du clavier,
\item \textbf{Spasmes} : Mouvement vers une case tous les \nbSecondes secondes,
\item \textbf{Dilatation} : Lenteur du personnage au maximum,
\item \textbf{Fureur} : Pose une bombe toutes les \nbSecondes secondes.
\end{itemize}