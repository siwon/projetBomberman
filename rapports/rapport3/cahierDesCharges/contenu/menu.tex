\subsection{Interfaces fournies}

Ce composant fournit deux interfaces aux autres modules :

\paragraph{IMenuToMain}

Cette interface permet de lier le composant à la méthode \textit{main} d'entrée du programme. Elle fournit la méthode \textbf{run} qui renvoit une des constantes suivantes : EXITGAME ou EXITERROR selon que le programme s'arrête sans ou avec un erreur.

\paragraph{IMenuToGameInterface}

Cette interface permet de lier le menu au composant d'interface de jeu lorsque celle-ci appelle le menu au cours d'une partie (lors d'une pause). La méthode \textbf{runPause} est alors appelée, renvoyant un signal à l'interface permettant de déterminer si le joueur reprend le jeu, ou quitte le programme. Le joueur pris en paramètre correspond à celui qui a appuyé sur le bouton pause.

\subsection{Interfaces requises}

Pour fonctionner, le composant Menu a besoin de 5 interfaces :
\begin{itemize}
    \item INetworkToMenu
    \item IControllerToMenu
    \item ISkin
    \item IConfigFile
    \item ISound    
\end{itemize}

\subsection{Structure du composant}

Le module est composé d'un tableau de \textit{IMenuScreen} pour chacun des menus proposés aux joueurs, qui s'appellent les uns les autres en renvoyant l'indice de l'écran suivant. Chaque écran est composé de plusieurs \textit{widgets} dessinables et actionnables spécialisés selon leurs fonctions.\\

Le diagramme de classes représentant ce composant est fourni en annexe
