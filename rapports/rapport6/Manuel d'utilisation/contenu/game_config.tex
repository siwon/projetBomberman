%Game config
	A partir du menu \textbf{Création d'une partie}, vous avez le choix entre créer une partie locale ou une partie en réseau.
Une fois le type de partie choisie, configurez le nombre de joueurs de la partie. 


Vous pouvez redéfinir les options de jeu dans le menu options, vous retrouverez les différentes significations des bonus de jeu dans l'annexe de ce manuel.


Une fois votre partie configurée, passez à l'étape suivante en validant. En fonction du type de partie choisie, veuillez vous réferrer à la section suivante correspondante.

	\subsubsection{Partie locale}
		L'étape suivante consiste à saisir les noms des différents joueurs de la partie. 
		
		Une fois les saisies validées, un résumé de la partie apparait avant le lancement de celle-ci.
		
	\subsubsection{Partie en réseau}
		L'étape suivante consiste à saisir le nombre de joueurs qui vont jouer sur votre ordinateur.
		Après validation, saisissez les noms des joueurs et validez.
		Un résumé de la partie va s'afficher, veuillez attendre la connexion de tous les autres joueurs avant de lancer la partie.
		
