A partir de ce menu, vous pouvez passer en mode plein-écran ou alors choisir votre skin personnalisé.
N'oubliez pas de valider vos changements avant de quitter ce menu.

\subsubsection{Créer son propre skin}
	
	Créez votre image d'arrière plan avec votre logiciel préféré. Cette image doit respecter les conditions suivantes : une taille de 800x600 pixels et au format png.\\
	
	Ouvrez un terminal, placez vous dans le répertoire d'installation et lancez l'utilitaire de création de skin par la commande suivante : \textbf{sh ./skin.sh}. Laissez vous guider par l'utilitaire. Si la création s'est bien déroulée, profitez de votre nouveau skin en le sélectionnant à partir du menu.