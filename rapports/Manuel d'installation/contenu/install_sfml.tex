
	Le programme nécessite la version 2.0 de la librairie multimédia SFML pour pouvoir s'exécuter correctement. Cependant cette version n'étant pas disponible sur les dépôts officiels ubuntu, il est nécessaire de l'installer manuellement. La procédure d'installation est décrite ci-dessus.
	
\subsection{Téléchargement de la librairie}

Téléchargez la librairie sur votre ordinateur en cliquant sur le lien suivant :
\href{https://github.com/LaurentGomila/SFML/tarball/master}{SFML2.0}


Si celui-ci n'est plus disponible, rendez-vous sur le site officiel \href{http://www.sfml-dev.org}{http://www.sfml-dev.org} et télécharger la version 2.0 présente dans la rubrique Télécharger.

\subsection{Installation des dépendances}

Ouvrez un terminal et mettez-vous en mode super-utilisateur en tapant la commande : \textbf{sudo su}.

Vous aurez besoin de l'utilitaire \textit{cmake} et de nombreuses dépendances afin de compiler la librairie. Procédez à l'installation de ceux-ci en tapant les commandes suivantes :
\begin{center}
	\begin{lstlisting}[caption={Installation des dépendances}]
		apt-get install cmake
		apt-get install libpthread-stubs0-dev
		apt-get install libgl1-mesa-dev
		apt-get install libx11-dev
		apt-get install libxrandr-dev
		apt-get install libfreetype6-dev
		apt-get install libglew1.5-dev
		apt-get install libjpeg8-dev
		apt-get install libsndfile1-dev
		apt-get install libopenal-dev
	\end{lstlisting}
\end{center}


\subsection{Compilation de la librairie}

Ouvrez de nouveau un terminal et mettez-vous en mode super-utilisateur en tapant la commande : \textbf{sudo su}.

Déplacez-vous dans le dossier de téléchargement de la librairie et extrayez le contenu du fichier via la commande : tar -xvf LaurentGomila-SFML-2.0-rc-24-gac9bda5.tar.gz SFML2.0

Placez-vous maintenant à la racine du dossier créé : cd SFML2.0

Puis tapez les commandes suivantes dans cet ordre là :

\begin{lstlisting}[caption={Compilation de la SFML}]
	cmake -G "Unix Makefiles" -D CMAKE_BUILD_TYPE=Release -D BUILD_SHARED_LIBS=TRUE .
	make
	make install
	cmake -G "Unix Makefiles" -D CMAKE_BUILD_TYPE=Debug -D BUILD_SHARED_LIBS=TRUE .
	make
	sudo make install
	cmake -G "Unix Makefiles" -D CMAKE_BUILD_TYPE=Release -D BUILD_SHARED_LIBS=FALSE .
	make
	sudo make install
	cmake -G "Unix Makefiles" -D CMAKE_BUILD_TYPE=Debug -D BUILD_SHARED_LIBS=FALSE .
	make
	sudo make install
\end{lstlisting}	

L'installation de la librairie est terminée.


